\documentclass{report}
\usepackage{homework}
\solstrue

\renewcommand{\hmwkTitle}{Homework 1}

\begin{document}
\mktitle

\begin{problem}
`traceroute' is a computer network diagnostic tool for displaying the route (path) and measuring transit delays of packets across an Internet Protocol (IP) network. In this problem, you will use the `traceroute' command to understand how packets route to a destination.

\begin{enumerate}
\item Run \textit{traceroute} command to find a route to `ucla.edu'. How many hops are there in between your local host to the destination? Copy and paste the result on your console in the answer box. (If you are using Windows Command Prompt, then use `\textit{tracert}' command instead.)
\item Run \textit{traceroute} command to find a route to `columbia.edu'. Copy and paste the result into the answer box.
\item Compare two results in terms of the number of hops and the delays.
\end{enumerate}

    \begin{answer}{45em}
    1. traceroute ucla.edu \newline
traceroute to ucla.edu (128.97.27.37), 64 hops max, 52 byte packets\newline
 1  wifi-131-179-34-2 (131.179.34.2)  19.077 ms  23.616 ms  28.306 ms\newline
 2  169.232.8.153 (169.232.8.153)  10.408 ms  3.762 ms  6.395 ms\newline
 3  cr00f2.csb1--sr02f2.csb1.ucla.net (169.232.8.7)  4.406 ms  6.125 ms  2.443 ms\newline
 4  128.97.27.37 (128.97.27.37)  3.877 ms !Z  3.150 ms !Z  2.922 ms !Z
\newline
\newline
 2. traceroute columbia.edu\newline
traceroute to columbia.edu (128.59.105.24), 64 hops max, 52 byte packets\newline
 1  192.168.0.1 (192.168.0.1)  1.918 ms  1.269 ms  3.888 ms\newline
 2  142.254.236.89 (142.254.236.89)  11.771 ms  14.149 ms  10.154 ms\newline
 3  agg57.snmncaby02h.socal.rr.com (76.167.30.5)  12.787 ms  12.506 ms  12.904 ms\newline
 4  agg20.lamrcadq02r.socal.rr.com (72.129.10.130)  17.546 ms  18.491 ms  20.633 ms\newline
 5  agg28.tustcaft01r.socal.rr.com (72.129.9.2)  15.655 ms  16.379 ms  15.727 ms\newline
 6  bu-ether26.tustca4200w-bcr00.tbone.rr.com (66.109.3.232)  20.543 ms  20.326 ms  15.965 ms\newline
 7  bu-ether14.lsancarc0yw-bcr00.tbone.rr.com (66.109.6.4)  18.607 ms
    be4.clmkohpe01r.midwest.rr.com (107.14.19.37)  18.848 ms  21.107 ms\newline
 8  66.109.5.123 (66.109.5.123)  15.894 ms  14.603 ms  15.061 ms\newline
 9  be5341.ccr41.lax04.atlas.cogentco.com (38.142.237.33)  14.400 ms  14.633 ms  16.071 ms\newline
10  be3360.ccr42.lax01.atlas.cogentco.com (154.54.25.149)  18.316 ms\newline
    be3271.ccr41.lax01.atlas.cogentco.com (154.54.42.101)  16.436 ms  16.115 ms\newline
11  be2932.ccr32.phx01.atlas.cogentco.com (154.54.45.161)  29.490 ms  27.842 ms\newline
    be2931.ccr31.phx01.atlas.cogentco.com (154.54.44.85)  28.326 ms\newline
12  be2930.ccr21.elp01.atlas.cogentco.com (154.54.42.78)  68.749 ms\newline
    be2929.ccr21.elp01.atlas.cogentco.com (154.54.42.66)  38.497 ms  36.803 ms\newline
13  be2928.ccr42.iah01.atlas.cogentco.com (154.54.30.161)  51.490 ms  52.323 ms  51.827 ms\newline
14  be2690.ccr42.atl01.atlas.cogentco.com (154.54.28.129)  66.612 ms  64.834 ms  66.018 ms\newline
15  be2113.ccr42.dca01.atlas.cogentco.com (154.54.24.221)  76.490 ms  76.683 ms  75.626 ms\newline
16  be2807.ccr42.jfk02.atlas.cogentco.com (154.54.40.109)  82.181 ms  83.560 ms  81.781 ms\newline
17  be2897.rcr24.jfk01.atlas.cogentco.com (154.54.84.214)  85.117 ms  83.047 ms  83.262 ms\newline
18  38.122.8.210 (38.122.8.210)  84.652 ms  83.747 ms  83.077 ms\newline
19  cc-core-1-x-nyser32-gw-1.net.columbia.edu (128.59.255.5)  91.146 ms  97.885 ms  84.604 ms\newline
20  cc-conc-1-x-cc-core-1.net.columbia.edu (128.59.255.210)  82.474 ms  83.668 ms  84.287 ms\newline
21  childpolicyintl.org (128.59.105.24)  82.140 ms  87.638 ms  84.382 ms\newline
\newline
3. For ucla.edu, since it's close to us, it has 4 hops and delay of 2ms, and columbia.edu is far from us, and it has 
21 hops and delay of 87 ms.
    
    \end{answer}

\end{problem}

\newpage


\begin{problem}
Host A is sending real-time voice over a packet-switched network. Host A converts analog voice to a digital 128 kbps bit stream on the fly. Host A then groups 1,600 bytes into a packet. Assume that the 1,600 bytes packet already includes all headers. There is one link between Hosts A and B; its transmission rate is 3 Mbps and its propagation delay is 20 msec. As soon as Host A gathers a packet, it sends it to Host B. As soon as Host B receives an entire packet, it converts the packet’s bits to an analog signal. How much time elapses from the time a bit is created (from the original analog signal at Host A) until the bit is decoded (as part of the analog signal at Host B)? In this problem, do not consider acknowledgement (response) from Host B.

    \begin{answer}{30em}
    Host A converts analog to digital bit stream 128 kbps \newline
    Host A grouping bits 1600-byte packets \newline
    Transmission rate 3mbps \newline
    Propagation delay 20 msec \newline
    Queuing delay: 1600 * 8/ 128 = 100 ms \newline
    Transmission delay: 1600 * 8 / 3000000 s = 4.267 ms \newline
    Time elapsed = Queuing delay + transmission delay + propagation delay = 100 + 4.267 + 20 = 124.267 ms
    
    \end{answer}

\end{problem}

\newpage


\begin{problem}
Tow hosts, A and B are separated by 20,000 kilometers and are connected by a direct link of $R=2Mbps$. Suppose the propagation speed over the link is $2.5*10^{8} meters/sec$.

\begin{enumerate}
\item Consider sending a file of 800,000 bits from Host A to Host B. Suppose the file is sent continuously as one large message. What is the maximum number of bits that will be in the link at any given time?
\item How long does it take to send the file, assuming it is sent continuously?
\item Suppose now the file is broken up into 20 packets with each packet containing 40,000 bits. Suppose that each packet is acknowledged by the receiver and the transmission time of an acknowledgment packet is negligible. Finally, assume that the sender cannot send a packet until the preceding one is acknowledged. How long does it take to send the file?
\end{enumerate}

    \begin{answer}{30em}
    1. Distance between A and B: 20 000 000 m\newline
    Transmission Rate R = 2Mbps = 2000 000 bps\newline
    Propagation Delay = distance / propagation speed = 20 000 000/250 000 000 = 0.08 sec\newline
    Max # of bits = Propagation Delay * R = 0.08 * 2000 000 = 16 0000 bits\newline \newline
    2. Transmission Delay = 800 000 / 2000 000 = 0.4 sec\newline
    total time = Propagation Delay + Transmission Delay = 0.08 + 0.004 = 0.48 sec\newline\newline
    3. Since the sender cannot send a packet until the preceding one is acknowledged,\newline
    The total time is 20 times the time for one packet\newline
    new Transmission Delay = 40 000 / 2000 000 = 0.02 sec\newline
    total time = 20 * (Propagation Delay + new Transmission Delay)\newline
    total time = 20 * (0.08 + 0.02) = 2 sec\newline
    \end{answer}

\end{problem}

\newpage


\begin{problem}

Alice and Bob are working remotely on a course project and are using \texttt{git} as the version control software.

\begin{enumerate}
\item Is it true that one must have GitHub/GitLab account to use git?
\item What is(are) the command(s) to initialize a local git repository?
\item Do Alice and Bob both must initialize local git repository?  If no, what are the alternative?
\item Let's consider that Alice modified the file \texttt{server.cpp}:
    \begin{enumerate}
    \item What git commands Alice needs to save modifications in the local git repository
    \item What git commands Alice needs to upload saved modifications to GitHub
    \item What git commands Bob needs to get Alice's changes and apply them to the local repository
    \end{enumerate}
\item Let's consider that both Alice and Bob modified the file \texttt{server.cpp} and Alice was first to successfully upload saved modifications (commit) to GitHub
  \begin{enumerate}
  \item Can Bob upload his changes without any additional actions? If no, why?
  \item If actions needed, list git commands that Bob will need to use to share his modifications with Alice.
  \end{enumerate}
\end{enumerate}

  \begin{answer}{37em}
  1. No. We just use `git' command in terminal. \newline\newline
  2. git init \newline
  git clone /path/to/repository \newline\newline
  3. No. They can work remotely: \newline
  git clone username@host:/path/to/repository \newline\newline
  4. (a) git add `filename' \newline
  git commit -m `Commit message' \newline
  (b) git push origin master 
  \newline
  If we have already cloned a new repository \newline
  (c) git fetch \newline
  git merge \newline\newline
  5. (a) No. He has to resolve the changes manually. Since the part he modified might be deleted by Alice in her latest change. Then use `git add' to commit the changes.
  \newline
  (b) git diff `source branch' `target branch' \newline
  Use the `diff' command to create a file showing the differences so Alice can know the modifications.\newline
  \end{answer}

\end{problem}

\newpage


\begin{problem}

You will learn some basic usages of \texttt{Vagrant} in your projects.

\begin{enumerate}
\item What is Vagrant mainly used for?
\item What is VirtualBox used for?
\item What is \textit{Vagrantfile}?
\item List at least five commands you can use with Vagrant.
\end{enumerate}

  \begin{answer}{35em}
  1. Vagrant is a tool for building and managing virtual machine environments in a single workflow. \newline \newline
  2. It sets up a virtual environment, that allows you to emulate an operating system on your own PC and use it like it's running on real hardware. \newline \newline
  3. Vagrantfile is a Ruby file used to configure Vagrant on a per-project basis. The main function of the Vagrantfile is to described the virtual machines required for a project as well as how to configure and provision these machines. \newline\newline
  4.vagrant init \newline
  vagrant up \newline
  vagrant ssh \newline
  vagrant halt \newline
  vagrant destroy
  \end{answer}

\end{problem}

\end{document}
